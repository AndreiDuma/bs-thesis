\chapter{Deploying and Using Lxchecker}
\label{chapter:deployment-and-usage}

This chapter covers all aspects of using Lxchecker. In \labelindexref{Section}{sec:deployment}, we present the process of installing and maintaining the platform from the perspective of faculty staff members. \labelindexref{Section}{sec:usage} shows how teachers prepare assignments and how students submit their solutions for grading. The approach is incremental -- we start with steps as easy as logging in and then proceed to more advanced procedures.


\section{Deploying Lxchecker}
\label{sec:deployment}

Lxchecker was designed with easy deployment and maintenance in mind. Given a cluster of Linux hosts, the only prerequisite for installing Lxchecker is Docker.

\subsection{Installing Lxchecker}
\label{sub-sec:install-lxchecker}

For this exposition, let's assume we want to deploy Lxchecker on four machines (either physical or virtual), named \texttt{host0} through \texttt{host3}. Docker is used as a deployment tool for all of Lxchecker's dependencies, so it needs to be installed on all hosts. The documentation\footnote{Installing Docker, \url{docs.docker.com/engine/installation}} provides easy setup instructions for various Linux distributions.

After installing Docker, we want to prepare the evaluation cluster -- let's use \texttt{host1}, \texttt{host2} and \texttt{host3}. We need to install\footnote{Build a Swarm cluster for production, \url{docs.docker.com/swarm/install-manual}} Swarm on all three machines, setup \texttt{host1} as a Swarm manager and add \texttt{host2} and \texttt{host3} to the cluster. From now, \texttt{host1} (the Swarm manager) will transparently behave as a giant Docker host and Lxchecker will talk to it for scheduling containers.

The other dependency is MongoDB. We will use \texttt{host0} for both the database server and Lxchecker itself. Docker provides an official image\footnote{\texttt{mongo} on Docker Hub, \url{hub.docker.com/_/mongo}} for MongoDB, which can be run by a simple command: \texttt{docker run -{}-name mongo-server -d mongo}.

Installing Lxchecker follows the same procedure, with two considerations: the container needs to be linked to \texttt{mongo-server} so it can reach it, and \texttt{host1}'s address must be specified via an environment variable. The resulting command: \texttt{docker run -{}-name lxchecker -{}-link mongo-server -e "DOCKER_HOST:host1" -d lxchecker/lxchecker}.

Navigating to \texttt{host0} in the browser should display Lxchecker's login page.

To ensure Lxchecker and its dependencies run continuously, Docker's built-in restart policies\footnote{Docker Restart Policies, \url{docs.docker.com/engine/admin/host_integration}} can be used. In case of a fatal error in one of the components, a restart policy will rerun the failed container to resume service.

\subsection{Updating Lxchecker}
\label{sub-sec:update-lxchecker}

Docker makes it easy to push a software update to production:
\begin{itemize}
	\item on the development machine, rebuild the \texttt{lxchecker/lxchecker} image from the updated code.
	\item push the new image to Docker Hub: \texttt{docker push lxchecker/lxchecker}.
	\item on \texttt{host0}, pull the new image from Docker Hub: \texttt{docker pull lxchecker/lxchecker}.
	\item reissue the \texttt{docker run} command.
\end{itemize}


\section{Using Lxchecker}
\label{sec:usage}

This section describes the user experience of students, teachers and admins working with Lxchecker. We first cover authentication, followed by subject \& assignment creation. Finally, we show how students upload code for testing and how their submissions are graded.

\subsection{Authentication}
\label{sub-sec:usage-authentication}

\fig[scale=0.5]{src/img/scr-log-in.pdf}{img:scr-log-in}{logging in}

When a user first visits the Lxchecker application, he is redirected to the login form displayed in \labelindexref{Figure}{img:scr-log-in}. If he provides valid credentials, he is logged in and showed the subject list in \labelindexref{Figure}{img:scr-subject-list}. If he doesn't have an account, he can click the "or \textbf{sign up}" link to visit the sign up page (see \labelindexref{Figure}{img:scr-sign-up}).

The subjects overview page in \labelindexref{Figure}{img:scr-subject-list} also includes a form for giving admin privileges. The username provided in the textbox must belong to an existing user.

\fig[scale=0.5]{src/img/scr-sign-up.pdf}{img:scr-sign-up}{creating a new account}

\subsection{Working With Subjects}
\label{sub-sec:usage-authentication}

In \labelindexref{Figure}{img:scr-subject-list}, users can see the list of all existing subjects. The forms in red (for adding subjects and admins) are only visible to admins. In the example picture, the \textbf{create subject} form is used to create an entry for the "Operating Systems" course. The \textbf{id} is short and meaningful, as it will be part of all future URLs in this subject.

\fig[scale=0.53]{src/img/scr-subject-list.pdf}{img:scr-subject-list}{available subjects list and \textit{create subject} form}

After submitting the \textbf{create subject} form, the admin is redirected to the new "Operating Systems" subject page (see \labelindexref{Figure}{img:scr-subject}). The sections in red are only visible to admins and teachers. Typically, an admin will only create a subject, then delegate permissions to a professor or teaching assistant in that course using the \textbf{add teacher} form. Once the teacher gains privileges in this subject, he can create assignments using the \textbf{create assignment} form.

\fig[scale=0.565]{src/img/scr-subject.pdf}{img:scr-subject}{subject page for the \textit{Operating Systems} course}

\subsection{Managing Assignments}
\label{sub-sec:usage-authentication}

Assignment parameters were explained in previous chapters. The example data filled out in \labelindexref{Figure}{img:scr-subject} is for an assignment on the topic of "Virtual Memory", using the \texttt{lxchecker/os-vmem} Docker image as a testing environment.

\fig[scale=0.563]{src/img/scr-assignment.pdf}{img:scr-assignment}{\textit{Virtual Memory} assignment page}

After submitting the \textbf{create assignment} form, the user is redirected to the "Virtual Memory" assignment page, displayed in \labelindexref{Figure}{img:scr-assignment}. A submission is already included in the figure to serve as an example.

The \textbf{assignment info} section contains information about:
\begin{itemize}
	\item \textbf{deadlines}: submissions will be penalized after the \textit{soft deadline} and will be considered "overdue" after the \textit{hard deadline}.
	\item \textbf{daily penalty} tells how many points will be subtracted from the submissions score for each day over the soft deadline.
	\item the \textbf{timeout} imposes a limit on the duration of the evaluation.
\end{itemize}

Section \textbf{my submissions} lists the user's own submissions, with information about evaluation status, points awarded by the system and whether the submission is \textit{active} (selected for final grading by the teacher). The current implementation automatically marks the most recent submission as \textit{active}.

\subsection{Submission Upload}
\label{sub-sec:usage-upload}

The \textbf{upload submission} form at the bottom of the page is used to submit a new file for evaluation. After clicking the \textbf{submit} button, the user is redirected to the new submission page.

\fig[scale=0.565]{src/img/scr-submission-pending.pdf}{img:scr-submission-pending}{waiting for the submission to be evaluated}

As shown in \labelindexref{Figure}{img:scr-submission-pending}, the submission is initially marked as \textbf{pending}. The \textbf{on time} label indicates the submission was uploaded before the soft deadline and no penalty is subtracted. Downloading the uploaded file is possible by using the included link. Because the evaluation has not finished, there are no logs yet.

\fig[scale=0.565]{src/img/scr-submission-done.pdf}{img:scr-submission-done}{submission was tested and score is displayed (full logs not included)}

Once the submission is tested, more information becomes available. The \textbf{metadata} field is populated and logs are fully displayed in the \textbf{execution logs} section. Note, however, that \textbf{grading status} is still \textit{not graded} until a teacher gives feedback and a score.

\subsection{Submission Grading}
\label{sub-sec:usage-grading}

\fig[scale=0.565]{src/img/scr-submission-grading.pdf}{img:scr-submission-grading}{grading form for teachers}

\labelindexref{Figure}{img:scr-submission-grading} shows the grading form. The teacher fills out the \textbf{score} and the textual \textbf{feedback}. After clicking the \textbf{submit} button, the \textbf{score by teacher} and \textbf{feedback} are populated and the student can see his final grade (score by tests + score by teacher - penalty). \labelindexref{Figure}{img:scr-submission-graded} displays the page of a graded submission.

\fig[scale=0.565]{src/img/scr-submission-graded.pdf}{img:scr-submission-graded}{submission after grading by teacher}
