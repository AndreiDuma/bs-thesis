\chapter{Conclusion}
\label{chapter:conclusion}

In this thesis we proposed \textbf{a system for automatically grading programming assignments}. Lxchecker is a web application for teachers and students, that allows code testing in a well-defined environment. It supports any number of subjects and assignments, features a role-based permission model and promises good performance.

In \labelindexref{Chapter}{chapter:intro}, we described what we want to build and what use cases we have in mind. We continued with presenting recent technical advancements in the field of virtualization and containerization. \labelindexref{Chapter}{chapter:sata-and-rw} also covers existing solutions for automatic grading in universities and online judges. Architectural design of Lxchecker was discussed in \labelindexref{Chapter}{chapter:architecture} -- we showed the high-level overview of the system and then detailed the purpose and interface of each component. In \labelindexref{Chapter}{chapter:implementation} we explained our choices of tools and libraries. We also investigated the implementation of the three main software modules, insisting on interesting problems we encountered and solutions we found. Because Lxchecker is designed to run in a server environment, \labelindexref{Chapter}{chapter:deployment-and-usage} gives instructions on how to install the system on a cluster of Linux machines. A second section describes the usage from the perspective of administrators, teachers and students, with an example scenario included. Finally, in \labelindexref{Chapter}{chapter:results} we present the current state of the project, listing both notable accomplishments and missing features. We also identify areas of improvement, proposing possible ideas and solutions.

We think Lxchecker will prove useful in academic institutions teaching Computer Science. Running the system in production will help us to better understand the needs of our users and allows us to further improve the functionality and behavior. We hope this paper will contribute to the development of superior products in the automated assignment grading market.
