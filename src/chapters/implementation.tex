\chapter{Implementation}
\label{chapter:implementation}

In this chapter, we will show how Lxchecker is implemented. Firstly, \labelindexref{Section}{sec:technologies} will describe the implementation technologies and explain why we use them. We then proceed to detail the inner workings of the three modules of the application. In \labelindexref{Section}{sec:scheduler-module}, we will see how the interaction with Docker is approached. \labelindexref{Section}{sec:storage-module} presents the API exposed by the storage interface module. After covering these dependencies, we will go over the web module in \labelindexref{Section}{sec:web-module}.


\section{Technologies}
\label{sec:technologies}

The quality of a software implementation greatly depends on the utilized tools. That is why we put a lot of thought into finding the suitable technologies for building Lxchecker. In this section we will first explain the general decision criteria we used. Each subsection will then cover a technology in detail, presenting its advantages, drawbacks and considered alternatives.

When contemplating an option, we analyzed the technology from the following perspectives. While each of them is important, we deemed some aspects more significant in the context of Lxchecker. In descending order of relevance:
\begin{itemize}
	\item \textbf{features}: we looked at what functionality we need and how well the particular technology provides them.
	\item \textbf{development and maintenance costs}: we wanted tools that make development easy, but also allow comfortable future maintenance. We searched for technologies that are easy to understand and use.
	\item \textbf{future support}: since Lxchecker aims to be a long-term solution for automated assignment grading, good backing from industry and community for its software dependencies is important.
	\item \textbf{performance}: both user experience and resource consumption are essential for Lxchecker.
\end{itemize}


\subsection{Docker}
\label{sec:tech-docker}

We already discussed the advantages of containers over virtual machines in \labelindexref{Chapter}{chapter:sata-and-rw}. Although using LXC (Linux Containers) directly was possible, our needs were better met by modern container engines. Among them, open-source Docker is by far the most popular. It is widely supported in all major cloud computing platforms (such as Amazon AWS\footnote{Amazon AWS Cloud, \url{aws.amazon.com}} or Google Cloud Platform\footnote{Google Cloud Platform, \url{cloud.google.com}}), has advanced orchestration and provisioning\footnote{Docker Compose, \url{docs.docker.com/compose/overview}} tools, features native clustering\footnote{Docker Swarm, \url{docs.docker.com/swarm/overview}}, and comes with a cloud-based image registry service\footnote{Docker Hub, \url{docs.docker.com/docker-hub}}. Docker has the most comprehensive set of features and is battle-tested by the industry, from which it has secured solid support.

An alternative to Docker is the more recent \texttt{rkt}\footnote{CoreOS's \texttt{rkt}, \url{coreos.com/rkt}} from CoreOS, which promises superior security and composability. Given \texttt{rkt}'s rather short presence on the market and because security is not crucial to our use case, we preferred the widely supported Docker.

\subsection{MongoDB}
\label{sec:tech-mongo}

For the persistence layer of Lxchecker, we chose to use MongoDB, a NoSQL database with dynamic schemas and intuitive JSON-like\footnote{BSON, \url{bsonspec.org}} document structure. MongoDB is open-source and provides drivers for a variety of programming languages\footnote{MongoDB Drivers, \url{docs.mongodb.com/ecosystem/drivers}}. It has good performance\footnote{Compare incomparable: PostgreSQL vs Mysql vs Mongodb, \url{http://erthalion.info/2015/12/29/json-benchmarks}} and built-in replication (useful for ensuring data redundancy). MongoDB has comprehensive documentation and strong community support.

\subsection{Go}
\label{sec:tech-go}

To glue dependencies together and implement the actual functionality, we decided to use Go\footnote{The Go programming language, \url{golang.org}} programming language. Modern yet widely adopted, Go is a compiled, statically typed language with garbage collection and CSP\footnote{Communicating sequential processes, \url{en.wikipedia.org/wiki/Communicating_sequential_processes}}-style concurrency features. It is appropriate for developing servers, balancing the performance and rigor of strongly-typed, compiled languages with the brevity of dynamically-typed languages. Other advantages include an extensive standard library and comprehensive documentation. In the recent years, Go has gained a lot of community traction, which, combined with the active support from companies like Google, suggests a stable future.

The two main alternatives considered were Java and Python. Java, although mature and efficient, promised a longer development time because of its verbosity. Python is at the other end of the brevity spectrum, producing simple and concise code, but its dynamic typing conflicted with the robustness we desired.

For working with Docker, we're using the official client library \texttt{engine-api}\footnote{\texttt{engine-api}, Go library for Docker, \url{godoc.org/github.com/docker/engine-api}}. It offers programmatic access to all of Docker's functionality as it available using the \texttt{docker} command line tool.

For MongoDB, we found the \texttt{mgo}\footnote{\texttt{mgo}, Go library for MongoDB, \url{godoc.org/gopkg.in/mgo.v2}} library very powerful. One feature that stands out is the ability to serialize and deserialize Go structs transparently, making the library almost as easy to use as an ORM framework.

In the web module, we are making heavy use of the standard library. All frontend templating is done using the \texttt{html/template} package and all request handling is managed by the standard \texttt{http} package. We decided against using a higher-lever web library to make future maintenance easier. Besides that, the standard library is powerful enough. We did use an external library for routing requests based on the URL path -- the \texttt{gorilla/mux}\footnote{\texttt{gorilla/mux}, request router for Go, \url{gorillatoolkit.org/pkg/mux}} package allows elegant routing schemas in a few lines of code.


\section{Scheduler Module}
\label{sec:scheduler-module}

This component acts as an interface between the web module and the Docker Swarm. It is implemented in the \texttt{scheduler} Go package. \labelindexref{Subsection}{sub-sec:scheduler-api} describes the simple API this package exposes, while \labelindexref{Subsection}{sub-sec:scheduler-execution} gives a step-by-step explanation of how evaluation scheduling works in LxChecker. 

\subsection{The API}
\label{sub-sec:scheduler-api}

The \texttt{scheduler} package provides a single \texttt{Submit} method that takes as an argument a \texttt{SubmitOptions} configuration object and returns a \texttt{SubmitResponse} (or an \texttt{error} in case of scheduling failure).

\labelindexref{Listing}{lst:submit-options} shows the \texttt{SubmitOptions} struct definition:
\begin{itemize}
	\item \texttt{Image} is the name of the Docker image that should be used as a testing environment (e.g. \texttt{lxchecker/ml_assignment_3})
	\item \texttt{Submission} contains the file uploaded by the user for evaluation -- it is copied to the container as instructed by the \texttt{SubmissionPath} field (see below)
	\item \texttt{SubmissionPath} tells the scheduler where in the container to put the submission, as expected by the tests in the image
	\item \texttt{Timeout} specifies for how long the scheduler should wait before forcefully killing the container and returning a \textit{time limit exceeded} error
\end{itemize}

\lstset{language=Golang,caption=the \texttt{SubmitOptions} struct,label=lst:submit-options}
\begin{lstlisting}
type SubmitOptions struct {
	Image          string
	Submission     []byte
	SubmissionPath string
	Timeout        time.Duration
}
\end{lstlisting}

The \texttt{SubmitResponse} struct definition is shown in \labelindexref{Listing}{lst:submit-response}.
\begin{itemize}
	\item \texttt{Logs} contain the combined \texttt{stdout} / \texttt{stderr} produced by the tests in the environment.
	\item \texttt{ExitCode} is the status code of the tests execution: 0 for a successful run (no fatal error that interrupted the evaluation), non-zero for error. Note that a successful run of the tests does not require tests to pass -- the actual testing results are extracted from the \texttt{Logs} based on \textit{metadata} (see below).
\end{itemize}

\lstset{language=Golang,caption=the \texttt{SubmitResponse} struct,label=lst:submit-response}
\begin{lstlisting}
type SubmitResponse struct {
	Logs     []byte
	ExitCode int
}
\end{lstlisting}

LxChecker uses \textbf{metadata} to allow environment designers (such as a teacher preparing an assignment) to expose data from the evaluation logs to the web frontend. For all log lines that match the format \textbf{\texttt{@key value}}, the frontend adds the corresponding key-value pair to a \textbf{metadata} map. A special required key is \textbf{\texttt{score}} (used for declaring the submission's evaluation score), but any number of keys can be defined. More on this topic will follow in \labelindexref{Section}{sec:web-module}.

\subsection{Execution Flow}
\label{sub-sec:scheduler-execution}

This subsection goes through the steps of running tests on a submission. An overview is presented in \labelindexref{Figure}{img:scheduler-overview}. The dotted box shows the separation between different hosts: the light gray box is part of the Docker cloud, while the dark gray rectangle represents the evaluation results on the host of the Lxchecker binary. Everything inside the dotted box happens on a Docker host selected by the Swarm manager.

\fig[scale=0.642]{src/img/scheduler-overview.pdf}{img:scheduler-overview}{execution steps performed by the scheduler}

When called, the \texttt{Submit} method goes through the following steps:
\begin{enumerate}
	\item \textbf{pull image}: the \texttt{Image} field in the \texttt{SubmitOptions} struct is used to fetch the appropriate testing environment image from the Docker Hub. This should happen only once on each Docker host, thanks to image caching.
	\item \textbf{create container}: the pulled image is used to create a new container (stopped for now).
	\item \textbf{copy submission}: the \texttt{Submission} bytes are tarred\footnote{tar archive, \url{en.wikipedia.org/wiki/Tar_(computing)}} and then extracted to \texttt{SubmissionPath} using Docker's \textit{copy-to-container} mechanism.
	\item \textbf{start container}: the container is started, invoking the test suite.
	\item \textbf{wait for finish or timeout}: the scheduler waits for the container to finish its execution. If the specified \texttt{Timeout} expires earlier, the container is stopped and an error is returned.
	\item \textbf{get results}: the \texttt{Logs} and \texttt{ExitCode} are extracted from the container and returned in a \texttt{SubmitResponse} struct.
\end{enumerate}


\section{Storage Interface Module}
\label{sec:storage-module}

This component is a high-level wrapper over the \texttt{mgo} library. It provides an ORM-like interface, with database models as first-class objects. Currently, there is support for reading, writing and updating entities.

\fig[scale=0.68]{src/img/database-schema.pdf}{img:database-schema}{database models and schema}

\labelindexref{Figure}{img:database-schema} illustrates database models and relationships between them. The fields in red represent primary keys. One more than one field in red is present, the key is composite. Each entity is stored in a separate MongoDB collection\footnote{MongoDB collection, \url{docs.mongodb.com/manual/reference/glossary/\#term-collection}}, indexed\footnote{Indexes in MongoDB, \url{docs.mongodb.com/manual/indexes}} by the primary key. The following paragraphs briefly present the purpose of each model and cover the most important fields.

\textbf{\texttt{Subject}} denotes a course for which automated grading is desired. The \texttt{Id} primary key must be a lowercase alphanumeric code (e.g. \texttt{ml} for a Machine Learning course), while \texttt{Name} should be the full name of the subject (e.g. "Machine Learning").

\textbf{\texttt{Assignment}} holds configuration data for testing environments. Assignments are grouped by subject, hence the parent link \texttt{SubjectId}. The \texttt{Id} should be a short, yet human-readable code for the assignment (e.g. \texttt{kmeans}). \texttt{Name} should be a lengthier description like "K-means Clustering". Fields \texttt{Image}, \texttt{Timeout} and \texttt{SubmissionPath} are used by the \texttt{scheduler} module, detailed in \labelindexref{Section}{sec:scheduler-module}. The remaining fields are necessary for implementing deadline support in the web module (see \labelindexref{Section}{sec:web-module}).

A \textbf{\texttt{Submission}} is created when a user uploads a file. The \texttt{AssignmentId} and \texttt{SubjectId} fields reference the parent assignment and together with \texttt{Id} serve as a composite primary key. \texttt{OwnerUsername} refers to the user that made the upload, while \texttt{Timestamp}, \texttt{UploadingFile} and \texttt{UploadingFileName} are traits of the upload itself. \texttt{Status} can be \texttt{pending}, \texttt{done} or \texttt{failed}. The next group of fields is populated on evaluation. The remaining fields are set when a teacher manually grades the submission, giving feedback.

The \textbf{\texttt{User}} refers to an account on Lxchecker. A user can log in using the stored \texttt{Username} and a password whose hash is \texttt{PasswordHash}. Platform administrators have the field \texttt{IsAdmin} set to true.

\textbf{\texttt{TeacherRole}} creates a many-to-many relationship between \textbf{\texttt{Users}} and \textbf{\texttt{Subjects}}. An entry in the \texttt{teacher_roles} MongoDB collection gives an user privileged read \& write access to an assignment.

\subsection{The API}
\label{sub-sec:db-api}

For each model described above, package \texttt{db} provides methods for:
\begin{itemize}
	\item \textbf{querying} objects:
		\begin{itemize}
			\item \texttt{Get*} methods returning exactly one object or an \texttt{db.ErrNotFound} error.
			\item \texttt{GetAll*} methods for returning a slice of object, possibly empty.
		\end{itemize}
	\item \textbf{inserting} objects:
		\begin{itemize}
			\item \texttt{Insert*} methods that return nothing when successful and \texttt{db.ErrNotFound} for missing parent objects or \texttt{db.ErrAlreadyExists} for duplicate insertions.
		\end{itemize}
	\item \textbf{updating} objects:
		\begin{itemize}
			\item \texttt{Update*} methods that return nothing when successful and \texttt{db.ErrNotFound} for non-existing keys.
		\end{itemize}
\end{itemize}

Support for object deletion is planned for the future. For now, the MongoDB interactive shell can be used by a privileged staff member.

\subsection{Implementation Example}
\label{sub-sec:db-implementation}

To illustrate how the \texttt{mgo} library is used for working with the MongoDB database, \labelindexref{Listing}{lst:get-submission} shows a simple query method from the \texttt{db} package.

\lstset{language=Golang,caption=GetSubmission query method,label=lst:get-submission}
\begin{lstlisting}
func GetSubmission(subjectId, assignmentId, id string) (*Submission, error) {
	submission := Submission{}
	c := mongo.DB("lxchecker").C("submissions")
	if err := c.Find(bson.M{
		"subject_id":    subjectId,
		"assignment_id": assignmentId,
		"id":            id,
	}).One(&submission); err != nil {
		if err == mgo.ErrNotFound {
			return nil, ErrNotFound
		}
		panic(err)
	}
	return &submission, nil
}
\end{lstlisting}


\section{Web Module}
\label{sec:web-module}

\todo{
	- package \texttt{web}
	- we will first discuss system ignoring authentication, then talk about package \texttt{util}
}

\subsection{Handlers}
\label{sub-sec:handlers}

\todo{
	- URL structure, error handling (panic recovery)
}

\subsection{Templates}
\label{sub-sec:templates}

\todo{
	- vanilla Go templates: package \texttt{html/template}
	- template inheritance schema, UI framework (Bootstrap), label-based system
}

\subsection{Authentication}
\label{sub-sec:authentication}

\todo{
	- states: unauthenticated, authenticated, +teacher, +admin
	- middleware, cookie format, permissions (roles), the RequestData struct
}
